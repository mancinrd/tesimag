\chapter{Plasma spectrum}
\label{ch:spectrometry}
%One of the fundamental characteristics of Plasma Coagulation Controller for medical applications is what species are produced and deposited during its application.
In various studies the spectrum of plasma DBD discharge in air at atmosferic pressure and ambient temperature has been observed (\cite{DBDair_Trot}, \cite{DBDAirTypicalSpec}). It presents peaks relative to reactive species from water, oxygen, nitrogen and its oxides in a wavelength range from $\num{200}$ to $\SI{880}{\nano\meter}$.

Reactive species that are thought to be involved in blood coagulation mechanisms are Reactive Oxidant Species (such as hydroxil radical \ce{OH}) and Reactive Nitrogen Species (derived from nitric oxide \ce{NO}) \cite{6153386}. This spectroscopy study is focalized on them and their precursor, i.e. on transitions relative to hydroxil, oxygen and molecular nitrogen.


\section{Optical Emission Spectroscopy}
The source produces plasma from a mixture gas of helium (or neon or argon) and air. Along free electrons, there are ions and species that collides with energetic electrons, and populate excited or metastable states with short lifetime. All reactive species partecipe in different reactions, and also in excitation and de-excitation reactions with conseguent emission of radiation. When an electron goes from state $p$ at higher energy to state $k$ of lower energy, radiation is emitted with central wavelenght $\lambda_0$. Power emitted by this radiation is given by radiant flux $d\phi_{\lambda}$ and selecting a solid angle as in figure \ref{fig:emissionfig}, is possible to define radiance $L_{\lambda}$, and intensity $I$, as in equations \ref{eq:emission}. Intensity for a radiation ultimately depends on $n(p)$, population density for state p, and Einstein Coefficient for the transition $A_{pk}$ that is typical for the transition \cite{book:291477}.
\begin{equation}
 \centering
 \begin{split}
 &\lambda_{0} = \frac{hc}{E_p - E_k} \\
 &L_{\lambda} = \frac{d^2\phi_{\lambda}}{dA \cos(\theta) d\Omega} \\
 &I = \int L_{\lambda} d\lambda = n(p) A_{pk}
 \end{split} 
 \label{eq:emission}
\end{equation}
\begin{figure}
 \centering
 \includegraphics[width = 0.45\textwidth]{Images/Spectroscopy/plasmaemission.png}
 \caption{Representation of plasma emission from a section of plasma with area $dA$, under a solid angle $d\Omega$ in a direction $\theta$.}
 \label{fig:emissionfig}
\end{figure}

Reactions that emit radiation in visible wavelenght in air are vibronic transitions where molecule goes from a vibrational state to another, with a change of vibrational quantum number $\nu$, and/or from a rotational state to another, with change of quantum number $J$ (\cite{book:137793}, \cite{wiki:vibronic}). When there is a vibrational transition, each line corresponds to different numbers $\nu'-\nu''$, these are transitions well spaced in the spectrum, easy to recognize. Rotational transitions gives birth to bands of peaks not markedly spaced, hard to resolve whitout an efficient spectrometer.

There are many reactions involving oxygen and nitrogen (see for example \cite{Kossyi_1992}), in this study only principal transition observable with our spectrometer are determined, to know dominant reactive species present in the plasma plume.

In the first section of this study plasma emission lines are recognized and studied for different discharge parameters. In the second section intensities relative to specific elements are observed for different positions along plasma plume and different gas compositions. 

\section{Line recognition}
Prototypes presented in chapter \ref{ch:electric} produce equivalent discharge conditions.
Plasma emission line recognition is done with prototype \textbf{A}, because it was the one that could be moved more easily to reach the spectrometer used in this part of the study.
A metal plate is positioned as target at a distance of $\SI{10}{\milli\meter}$ from plasma exit, as in figure \ref{fig:app1}. Helium with a flow set to $\SI{2}{\liter/\minute}$ is used to start the discharge.

The spectrometer is an IsoPlane that separates emissions with different wavelenghts with a grating. The spectrometer has a focal lenght of $\SI{320}{\milli\meter}$ and is equipped with three different gratings: $\num{150}$, $\num{1200}$ and \SI{2400}{gg/\milli\meter}, corresponding to different resolutions. %of $\num{0.26}$, $\num{0.03}$ and $\SI{0.01}{\nano\meter}$.
Light emitted by plasma is collected through a quartz lens, on the left in  figure \ref{fig:app1},%...
and travels in an optical fiber %..
connected to the spectrometer entry. At the spectrometer exit there is a CCD camera of $2048$ pixels and a count limit of $\num{65000}$.
\begin{figure}
\centering
\includegraphics[width=.6\textwidth]{Images/Spectroscopy/apparato.jpg}
\caption{Picture of the experimental setup for line recognitizion. It is possible to see the source, the metal target and the optical setup on the left. Plasma emission is collected by the lens and sent to the spectrometer.}
\label{fig:app1}
\end{figure}

Once a grating is chosen, it is possible to set the starting wavelength on the acquisition system and from there the spectrometer takes measures until the end of the CCD, resulting in a specific wavelength interval for each grating.
For each measure we select an appropriate acquisition time that allows to observe peaks with a good count number and avoid saturation.

It's important to stress out that, with this measuring method and due to complexity of plasma reactions and composition, it's not possible to extrapolate quantitative considerations between different species concentration. However it's possible to recognize the presence of certain species and make some considerations watching spectra variation with different experimental setup.


\subsection{Emission measurements}
The emissions spectrum is observed in a wavelength range from $\num{230}$ to $\SI{800}{\nano\meter}$, with standard discharge parameters: $f \SI{5}{\kilo\hertz}$ and $V_{p} = \SI{6.0}{\kilo\volt}$.
The first measurs is a rapid acquisition with the lowest resolution possible, to see intersting regions and have an idea of required exposition times. After that, the spectrometer does another acquisition with higher resolution for all wavelengths, measuring several spectra. The entire spectrum is reconstructed attaching different spectra, showed in figure \ref{fig:spectr}, where are labelled principal transitions.
For each measure a background spectrum si also taken, without plasma, to recognize peaks that are not relative to plasma emission.

The spectra are read with IDL routines \cite{GUMLEY200215} and analyzed with ROOT \emph{TSpectrum.h} library \cite{ROOT:TSpectrum}.
In each spectrum each channel is divided by the exposition time of the spectrum, evaluating the count rate at a specific wavelength.
White noise contribution is estimated as the average value from a portion of the spectrum that does not presents peaks, hence it is subtracted to count rates for each wavelength.
In the resulting spectra emission peaks are found with TSpectrum functions (where is possible to set a treshold in heigth and the general width for lines to be searched) and isolate peaks from background. The exact wavelength for each transition is found with a gaussian fit in an interval that takes into consideration the asymmetry of the peak where it's needed.

As said before, this study is focused on measure related to ROS and RNS, so in lines for \ce{NO}, \ce{OH} and \ce{N_2}.
\begin{figure}
\centering
\includegraphics[width=0.99\textwidth]{Images/Spectroscopy/spettrotot_unico_label_def.png}
\caption{Spectrum with helium flow of $\SI{2}{\liter/\minute}$, pulse parameters of $f = \SI{5}{\kilo\hertz}$ and $V_p = \SI{6.0}{\kilo\volt}$. The emission is collected at the end of the nozzle, where plasma exits in air.}
\label{fig:spectr}
\end{figure}


\paragraph{\ce{NO} lines}
In the spectrum there are two doublets for the transition $\ce{A^2\Sigma^+} \rightarrow \ce{X^2\Pi}$ with vibrational numbers (0-0) and (0-1) (\cite{Knie:166349}, \cite{VANSPRANG197955}), presented in table \ref{tab:spettroNO}. Intensities for the peaks are normalized with maximum value of $\num{1000}$ for the acquisition, the table shows as the intensities for this transition is very low. Other transitions relative to this molecule have even lower relative intensity and are not observed.
\begin{table}[h]
\centering
 \begin{tabular}{cc}
  \toprule
  $\lambda$ \text{[}\si{\nano\meter}\text{]} &\text{I [arb.u.]}\\
  \midrule
  \num{236.31(24)}  &27\\
  \num{237.00(15)}  &26\\
  \num{247.02(5)}  &28\\
  \num{247.86(12)}  &27\\
  \bottomrule
 \end{tabular}
 \caption{Peaks measured for \ce{NO}.}
 \label{tab:spettroNO}
\end{table}


\paragraph{\ce{OH} lines}
In the spectrum is observed the rotational band for transition (\ce{A^2\Sigma}, $\nu' = 0$ $\rightarrow$ \ce{X^2\Pi}, $\nu'' = 0$), $13$ lines are found \cite{doi:10.1142/S0129183100000857}.
In figure \ref{fig:OHsp} a portion of the spectrum in intersting wavelength range for \ce{OH}, in table \ref{tab:sptrOH} are presented peak values.
\begin{figure}
 \centering
 \includegraphics[width=0.6\textwidth]{Images/Spectroscopy/OH_f5t16v_2.png}
 \caption{Emission spectrum of helium discharge with $f = \SI{5}{\kilo\hertz}$ and $V_p = \SI{6.0}{\kilo\volt}$ from \num{306} to \SI{310}{\nano\meter}. It is possible to see the rotational band of \ce{OH}.}
 \label{fig:OHsp}
\end{figure}
\begin{table}
 \centering
 \begin{tabular}{cc}
  \toprule
  $\lambda$ \text{[}\si{\nano\meter}\text{]} &\text{I [arb.u.]}\\
  \midrule
  \num{306.96(1)}  &53\\
  \num{307.11(1)}  &58\\
  \num{307.29(1)}  &62\\
  \midrule                          
  \num{307.94(1)}  &142\\
  \num{308.09(1)}  &148\\
  \num{308.26(1)}  &161\\
  \num{308.43(1)}  &112\\
  \num{308.62(1)}  &46\\
  \num{308.74(1)}  &137\\
  \midrule                          
  \num{309.11(1)}  &151\\
  \num{309.22(1)}  &120\\
  \num{309.45(1)}  &36\\
  \num{309.73(1)}  &125\\
  \bottomrule
 \end{tabular}
 \caption{Peaks position and intensity measured for \ce{OH} in helium discharge with parameters $f = \SI{5}{\kilo\hertz}$, $V_p = \SI{6.0}{\kilo\volt}$.}
 \label{tab:sptrOH}
\end{table}


\paragraph{\ce{N_2} and \ce{N+_2} lines}
Measured spectrum presents several lines relative to nitrogen molecules, including the strongest, for diatomic molecule dinitrogen. The Second Positive System for \ce{N2} transition $\ce{C^3\Pi} \rightarrow \ce{B^3\Pi}$ and the First Negative System for \ce{N2+} transition $\ce{B^2\Sigma} \rightarrow \ce{X^2\Sigma}$ are observed. Their positions and peak values are presented in table \ref{tab:sptrN} (\cite{N2lab}, \cite{Britun_2007}). For \ce{N2} also a band of multiple rotational lines centered around \SI{336.58(1)}{\nano\meter} is observed.
Some of the peaks are seen in the second diffraction order, where there is more distance between lines. In figure \ref{fig:N2} are presented two portions of the spectrum showing \ce{N2} lines.
\begin{figure}
 \centering
 \subfloat[Transitions with $\Delta \nu = 2$]{
    \includegraphics[width=0.45\textwidth]{Images/Spectroscopy/N2v_f5t16_2.png}
 }
 \hfill
 \subfloat[Strongest line (0-0), $2^{\text{nd}}$ diffraction order]{
    \includegraphics[width=0.45\textwidth]{Images/Spectroscopy/N2r_f5t16_2.png}
 }
 \caption{Emission spectrum of helium discharge with $f = \SI{5}{\kilo\hertz}$ and $V_p = \SI{6.0}{\kilo\volt}$ from \num{365} to \SI{382}{\nano\meter} (a) and from \num{670} to \SI{675}{\nano\meter} (b). It is possible to see vibrational and rotational bands for \ce{N_2} transitions.}
 \label{fig:N2}
\end{figure}

\begin{table}
\centering
 \begin{tabular}{cccc}
  \toprule
                            &$\lambda$ \text{[}\si{\nano\meter}\text{]} &\text{I [arb.u.]}  &($\nu'-\nu''$)\\
  \midrule
  \multirow{3}*{\ce{N_2}}   &\num{316.03(1)}  &381  &(1-0)\\
                            &\num{337.11(1)}  &1000 &(0-0)\\
                            &\num{357.77(1)}  &722  &(0-1)\\
  \midrule
  \multirow{4}*{\ce{N_2}}   &\num{367.22(20)}  &58  &(3-5)\\
                            &\num{371.12(4)}  &172  &(2-4)\\
                            &\num{375.66(2)}  &232  &(1-3)\\
                            &\num{380.64(2)}  &423  &(0-2)\\
  \midrule
  \multirow{2}*{\ce{N_2^+}} &\num{391.50(2)}  &355  &(0-0)\\
                            &\num{427.45(2)}  &180  &(0-1)\\
  \bottomrule
 \end{tabular}
 \caption{Peaks position and intensity measured for \ce{N_2} and \ce{N+_2} in helium discharge with parameters $f = \SI{5}{\kilo\hertz}$, $V_p = \SI{6.0}{\kilo\volt}$.}
 \label{tab:sptrN}
\end{table}


\paragraph{Atomic lines}
Other lines relative to different elements are presented in table \ref{tab:sptrother}, they are \cite{NIST}:
\begin{itemize}
 \item \textbf{\ce{H_{\alpha}}} line corresponding to transition from quantum number $n=3$ to $n=2$
 \item \textbf{\ce{He}} two of the strongest lines for helium
 \item \textbf{\ce{O}} strong line of oxygen
\end{itemize}
\begin{table}
\centering
 \begin{tabular}{ccc}
  \toprule
                            &$\lambda$ \text{[}\si{\nano\meter}\text{]} &\text{I [arb.u.]}\\
  \midrule
  \ce{H_{\alpha}}           &\num{655.96(4)}  &113\\
  \midrule
  \multirow{2}*{\ce{He}}    &\num{586.94(5)}  &122\\
                            &\num{705.56(1)}  &649\\
  \midrule
  \ce{O}                    &\num{776.89(1)}  &393\\
  \bottomrule
 \end{tabular}
 \caption{Peaks position and intensity measured for \ce{H}, \ce{He} and \ce{O} in helium discharge with parameters $f = \SI{5}{\kilo\hertz}$, $V_p = \SI{6.0}{\kilo\volt}$.}
 \label{tab:sptrother}
\end{table}


\subsection{Pulse settings}
Plasma emission intensity is measured with different voltage peak values and pulse repetition rates to understand how discharge settings influence reactive species production.

As seen in chapter \ref{ch:electric} for different pulse repetition rates the electric behavior stays constant, however this parameter could still influence species production rate. Pulse repetition rate is related to the energy deposited on the ionized gas, an higher rate implies more pulses in a given time.

%Reactions that produce and recombine reactive species, and consequently density and lifetime of species, are influenced by electric field and duration of the discharge.
Emission spectrum is observed with three different parameter combinations, corresponding to different intensity of the treatment:
\begin{itemize}
 \item low: $f = \SI{5}{\kilo\hertz}$ and $\Delta t = \SI{15}{\micro\second}$
 \item medium: $f = \SI{10}{\kilo\hertz}$ and $\Delta t = \SI{10}{\micro\second}$
 \item high : $f = \SI{15}{\kilo\hertz}$ and $\Delta t = \SI{10}{\micro\second}$
\end{itemize}

For each setting plasma emission is observed along two different lines of sight:
\begin{itemize}
 \item position 1: as close as possible to the end of the nozzle, near plasma exit point from the source;
 \item position 2: close to the target, at \SI{10}{\milli\meter} from plasma exit point, where it collides with target.
\end{itemize}

Intensities for \ce{OH} and \ce{N2} species are evaluated collectively for the lines in a specific wavelength range. For \ce{OH} lines all the rotational band between $\num{306}$-\SI{309}{\nano\meter} is considered. Lines relative to \ce{N2} are separated in those between $\num{335}$-\SI{337}{\nano\meter} (rotational band and (0-0) transition) and those between $\num{368}$-\SI{382}{\nano\meter} (vibrational transitions with $\Delta \nu = 2$).

In figure \ref{fig:irel} there are measurement results for considered lines.

Intensities for \ce{OH} decreases drastically increasing the distance from the source, in position 2 values are lower than $0.1\%$ of those from position 1. \ce{OH} lines for both positions have same intensity with low and medium power setup, while intensity is lower with higher frequency. For both positions there is a similar behavior.

Also \ce{N2} intensities are related to pulse repetition rate: they decrease as the pulse rate increases. Increasing the rate from \num{5} to \SI{15}{\kilo\hertz} intensity values decrease around $60\%$ in position 1, and reach lower values for position 2.

It seems that production of both those reactive species have rates dependant on pulse repetition rates, in particular the intensity of their emission decreases at higher frequencies.
\begin{figure}
\centering
 \subfloat[\ce{OH} intensities in range $\num{306}$-\SI{309}{\nano\meter}]{
    \includegraphics[width=0.48\textwidth]{Images/Spectroscopy/If_OH.png}
 }
 \hspace{0.55\textwidth}
 \subfloat[\ce{N2} intensities range $\num{335}$-\SI{337}{\nano\meter}]{
    \includegraphics[width=0.48\textwidth]{Images/Spectroscopy/If_N2r.png}
 }
 \hfill
 \subfloat[\ce{N2} intensities range $\num{368}$-\SI{382}{\nano\meter}]{
    \includegraphics[width=0.48\textwidth]{Images/Spectroscopy/If_N2v.png}
 }
 \caption{Relative intensities of selected portions of the spectrum, for different pulse repetition rates, for position 1 and position 2.}
 \label{fig:irel}
\end{figure}


\subsection{Estimation of plasma temperatures}
From diatomic molecule's spectra it's possible to evaluate some parameters that are indicators of plasma's state: rotational temperatures for \ce{OH} and \ce{N2}, $T_{r}$, and vibrational temperature for \ce{N2}, $T_{v}$.
These parameters are estimation of the temperature at which thermal energy is comparable to the gap energy between rotational or vibrational state transitions, they can be defined as in equations \ref{eq:temperatures} where $\nu$ is the vibrational quantum number and $I$ is the quantizied moment of inertia of the molecule.
\begin{equation}
 \centering
 \begin{split}
  &T_{r} = \frac{\hslash^2}{2 k_{B} I}\\
  &T_{v} = \frac{h \nu}{k_{B}}
 \end{split}
 \label{eq:temperatures}
\end{equation}

Rotational temperatures can be considered an estimation of neutral gas kinetic temperature. Vibrational temperature gives an idea of the population of vibrational states, useful to determine chemical reactions inside plasma.

\subsubsection{Rotational temperature for \ce{OH} and for \ce{N2}}
In rotational bands the intensity of a transition for a specific wavelength is proportional to the number density popolation of upper state (equation \ref{eq:emission}), that, considering a Maxwell-Boltzmann distribution, is proportional to the temperature of the species. In equation \ref{eq:Irot} the proportionality is explicitated, with $D_{0}$ parameter that depends on number of initial molecules, partition function of the rotational state and quantum rotational numbers for upper and lower state, $S$ is the oscillator strenght specific for the molecule and $E_{r}$ depends from a constant defined by the vibrational state and from quantum rotational number for upper state \cite{MOON2003249}.
\begin{equation}
 \centering
 I = \left(\frac{2 \pi}{\lambda}\right)^4 D_0 S \exp\left(-\frac{E_r}{k_B T_r}\right)
 \label{eq:Irot}
\end{equation}

As explained before, rotational bands have many lines, not all distinguishable with the spectrometer used in this work. An approach to temperature estimation is to simulate spectra with different temperatures and to minimize differences for measured spectrum (\cite{Izarra_2000}).
In a predetermined range of temperatures spectra with different temperatures are simulated, where each line is a gaussian peak with its width that takes into consideration broadning due to thermal motion, Doppler effect and measure resolution. For each spectrum the mean square difference is evaluated and the temperature associated with the minimum difference is chosen as the one relative to the spectrum. The error is estimated taking an upper and a lower limit where the mean square difference is larger than $5\%$ of the minimum value. An example of the spectrum is shown in figure \ref{fig:Trfit}, while resulting temperatures are shown in figure \ref{fig:Trval}.
\begin{figure}
\centering
  \subfloat[\ce{OH}]{
    \includegraphics[width=0.48\textwidth]{Images/Spectroscopy/OHFit_f5t16elio.png}
  }
  \hfill
  \subfloat[\ce{N_2}]{
    \includegraphics[width=0.48\textwidth]{Images/Spectroscopy/N2rotFit_elio.png}
  }
 \caption{Example of optimal spectrum simulation for \ce{OH} and \ce{N_2} considered species.}
 \label{fig:Trfit}
\end{figure}


Estimated temperatures are compatible with each other, for each distance and for each pulse setup. It's then possible to evaluate a mean value for the species, that are $T_{r,\ce{OH}} = \SI{352(38)}{\kelvin}$ and $T_{r,\ce{N2}} = \SI{321(41)}{\kelvin}$, compatible with each other, and that, as said before, can be taken as an indicator of kinetic temperature for neutral species, so as temperature of the fluid. Those temperatures are higher then room temperature, but they are compatible with the definition of cold plasma.
\begin{figure}
\centering
  \subfloat[\ce{OH}]{
    \includegraphics[width=0.48\textwidth]{Images/Spectroscopy/TrOH2.png}
  }
  \hfill
  \subfloat[\ce{N2}]{
    \includegraphics[width=0.48\textwidth]{Images/Spectroscopy/TrN2_2.png}
  }
 \caption{Estimation of rotational temperature of \ce{OH} and \ce{N2} molecules, for different parameters setup and positions.}
 \label{fig:Trval}
\end{figure}

\subsubsection{Vibrational temperature for \ce{N_2}}
Given a set of vibrational transition lines with defined $\Delta \nu = \nu' - \nu''$, their relative intensities are correlated to each other, with a proportionality that involves vibrational temperature \cite{Britun_2007}.
With peak intensities estimated for a given transition can be made the Boltzmann graph shown in figure \ref{fig:TvN2} (a), for each experimental condition. From the Boltzmann graphs $T_v$ can be evaluated as in formula \ref{eq:Tv}.
\begin{equation}
 \begin{split}
  & log(I(\nu') = S \nu'' + I_0 \\
  & T_v [K] = \frac{10^4}{3.57 \cdot S - 0.03}
 \end{split}
\label{eq:Tv}
\end{equation}

Results are in figure \ref{fig:TvN2}. For this parameter values are compatible with each other at low and medium power pulse settings with a mean value of $T_v = \SI{3405(154)}{\kelvin}$, while a lower temperature is found for high power pulse settings $T_v = \SI{2781(322)}{\kelvin}$. It seems that whith an higher pulse repetition rate there is lower concentration of excited \ce{N_2}, produced with lower energy.
\begin{figure}
\centering
  \subfloat[Boltzmann graph for discharge with $f = \SI{5}{\kilo\hertz}$, $\Delta t = \SI{16}{\micro\second}$]{
    \includegraphics[width=0.48\textwidth]{Images/Spectroscopy/boltzman_f5t16v_2.png}
  }
  \hfill
  \subfloat[vibrational temperature of \ce{N2}]{
    \includegraphics[width=0.48\textwidth]{Images/Spectroscopy/TvN2_2.png}
  }
  \caption{(a) Boltzmann graph made from \ce{N_2} vibrational transition intensities for helium discharge with $f = \SI{5}{\kilo\hertz}$ and $V_p = \SI{6.0}{\kilo\volt}$, in position 1; (b) vibrational temperatures of \ce{N2} molecule for different pulse paramters settings and emission measured from position 1 and 2.}
  \label{fig:TvN2}
\end{figure}


\section{Line intensity analysis}
The second phase of this study is dedicated to measurements of plasma emission intensiy at different positions along the plasma plume and with different gas composition.

For this study are used a portable mini spectrometer and source prototype \textbf{B}, with an experimental setup similar to the one in chapter \ref{ch:shape}. 

\subsection{Experimental setup}
In figure \ref{fig:app2} can be seen the source and the lens focalized where plasma forms, mounted on a support with variable height. The source mounts the glass nozzle used in chapter \ref{ch:shape}, allowing to observe emission also inside it. A conductive target is placed at \SI{10}{\milli\meter} from nozzle end.
\begin{figure}
\centering
\includegraphics[width=.7\textwidth]{Images/Spectroscopy/app2_lines.png}
\caption{Picture of the experimental setup for intensity measurements. Spectrometer's lens on the left is focalized on the plasma plume, mounted on a support with variable height.}
\label{fig:app2}
\end{figure}

It is used a mini spectrometer \emph{Hamamatsu C10082CAH}, with resolution of \SI{1}{\nano\meter}, spectral range from \SI{150}{\nano\meter} to \SI{850}{\nano\meter} and a CCD sensor \emph{S10420-1106} with \num{2048} pixels. The efficiency of the spectrometer for different wavelengths is evaluated with measurements of emission from a lamp with known emission spectrum. With this calibration is possible to convert the number of counts in the number of photons emitted by plasma for each wavelength.

Helium, neon or argon are used to produce plasma, spectra from those three gasses is shown in figure \ref{fig:specs}. Given the low resolution of this spectrometer, it is not possible to identify all the peaks that are observed. Each spectrum presents the same emission lines and, in addition, lines relative to the element used to start the discharge. It is possible to identify four wavelength ranges relative to four elements:
\begin{itemize}
 \item \ce{N_2} lines from \num{320} to \SI{480}{\nano\meter}: one singlet and several doublets relative to the SPS described before in this chapter; 
 \item one \ce{He} line at \SI{706.6}{\nano\meter}, observed only for helium plasma;
 \item \ce{Ne} lines from \num{410} to \SI{780}{\nano\meter}, several lines observed only for neon plasma with intensity generally higher than \ce{N_2} lines;
 \item \ce{Ar} lines from \num{660} to \SI{850}{\nano\meter}, several lines observed only for argon plasma.
\end{itemize}
\begin{figure}[h]
 \centering
 \subfloat[Spectrum with helium gas.]{
    \includegraphics[width=0.65\textwidth]{Images/Spectroscopy/B_elio_sp.png}
  }
  
  \subfloat[Spectrum with neon gas.]{
    \includegraphics[width=0.65\textwidth]{Images/Spectroscopy/B_neon_sp.png}
  }
  
  \subfloat[Spectrum with argon gas.]{
    \includegraphics[width=0.65\textwidth]{Images/Spectroscopy/B_argon_sp.png}
  }
  \caption{Plasma emission spectrum measured at the end of the nozzle, with different gasses used to start the discharge.}
  \label{fig:specs}
\end{figure}


Emission for different elements is evaluated selecting, for each element, some of the higher peaks in all spectra between those observed:
\begin{itemize}
 %\item UVB lines : doublet centered at \SI{293.2}{\nano\meter} and \SI{295.5}{\nano\meter} that could be emitted by \ce{NO} molecules, but we can not be sure due to spectrometer's low resolution;
 \item \ce{N_2} : three lines centered at \SI{353.6}{\nano\meter} (transition $(1-2)$), \SI{380.5}{\nano\meter} (transition $(0-2)$) and \SI{405.9}{\nano\meter} (transition $(0-3)$);
 \item \ce{He} : one line at \SI{706.6}{\nano\meter};
 \item \ce{Ne} : one doublet at \SI{638.3}{\nano\meter} and \SI{640.6}{\nano\meter}, a single line at \SI{703.5}{\nano\meter} and a single line at \SI{725.5}{\nano\meter};
 \item \ce{Ar} : three different lines at \SI{696.5}{\nano\meter}, \SI{763.5}{\nano\meter} and \SI{811.5}{\nano\meter}.
\end{itemize}


\subsection{Position}
The experimental setup allows to measure plasma emission at different heights along the nozzle axis, emission intensity evaluated in four different positions:
\begin{itemize}
 \item \SI{-5}{\milli\meter}, inside the nozzle
 \item \SI{0}{\milli\meter}, at the end of the nozzle
 \item \SI{5}{\milli\meter}, between nozzle and target
 \item \SI{10}{\milli\meter}, right before target position
\end{itemize}

Results are in figure \ref{fig:Irel_pos}. Total intensities are normalized to compare emissivity variation for different positions, while relative intensities are divided for the respective total intensity.
With helium emission is lower inside the nozzle, increase at the exit and decreases slightly moving towards the target. For other gasses emission decreases increasing distance from the source.
For every gas, emission from \ce{N_2} increases until it reaches a constant value.
Lines from specific gasses decreases distant from the nozzle, with slope different for different gasses.
It's intersting to note how neon emission decreases linearly, while \ce{N_2} emission increases and it's an effect that is observed visually on the plume: going from the nozzle to the target, plasma colour goes to red (neon emission) to violet (nitrogen emission).
\begin{figure}
\centering
  \subfloat[Total emission intensity.]{
    \includegraphics[width=0.48\textwidth]{Images/Spectroscopy/Itot_pos2.png}
  }
  
  %\subfloat[UVB lines]{
  %  \includegraphics[width=0.48\textwidth]{Images/Spectroscopy/Irel_NO_pos.png}
  %}
  
  \subfloat[\ce{N_2} lines]{
    \includegraphics[width=0.48\textwidth]{Images/Spectroscopy/Irel_N2_pos2.png}
  }
  \hfill
  \subfloat[\ce{He} lines, \ce{Ne} lines, \ce{Ar} lines.]{
    \includegraphics[width=0.48\textwidth]{Images/Spectroscopy/Irel_gas_pos2.png}
  }
\caption{Axial profile of total intensities (a) and relative intensities for selected portions of the spectrum (b-c), with different starting gas. At \SI{0}{\milli\meter} the lens points at the end of the nozzle, at \SI{10}{\milli\meter} at a metal target. Relative intensities in (c) are for lines corresponding to the element that used as starting gas. Relative intensities take into consideration spectrometer's efficiency and total emission for each position.}
 \label{fig:Irel_pos}
\end{figure}

It can be concluded that inside the nozzle the emission depends from the gas used to start the plasma, going outside the emission due to nitrogen species becames dominant in the discharge.


\subsection{Gas composition}
Relative intensity for selected lines is analyzed for different gas composition. Helium is mixed with argon or nitrogen using specific flowmeters, that allow to add up to \SI{0.2}{\liter/\minute}, with resolution of \SI{0.01}{\liter/\minute}.
With an helium flow to \SI{2}{\liter/\minute} is possible to have gas mixtures where argon and nitrogen have a maximum percentage of $\num{10}\%$.

The spectrometer is pointed at the end of the nozzle and collects spectra for three different gas concentrations, measuring intensities shown in figure \ref{fig:Irel_flow}.

When nitrogen is added to the gas, total emission intensity lowers, while nitrogen emission does not change. It is possible that the increase in nitrogen concentration is too low if compared to the quantity naturally present in air and consequently there is not a significative increase of nitrogen emission at the end of the nozzle.
In this position the gas is a mix between the gas used to start the discharge and air, where nitrogen is approximately the $70\%$ of air. The percentage of gas in air is a function of the gas flow, for example it is possible to hypotize that at the end of the nozzle the percentage of discharge gas and air is equal. With a gas flow of pure helium, nitrogen would be the $35\%$ of the total gas density. If the helium flow is substituted with $10\%$ of nitrogen and $90\%$ helium, as in the experiment condition, nitrogen would be the $40\%$ of total gas density. If the relative emission of nitrogen is proportional to the fraction of nitrogen in the gas, the increase in the emission would be a value around $5\%$, an increment that could be not resolvable with the apparatus used in this experiment. Further measurements of nitrogen emission could verify this hypothesis and, in that case, they could be used to estimate nitrogen percentage for different positions inside the nozzle. 


When argon is added to helium, total emission intensity increases, while relative emission from elements other than argon decreases slightly. It means that the only variation when there is a percentage of argon is that the emission relative to this element adds t othe emission of helium and the emission relative to other elements stays unchanged.
\begin{figure}
\centering
  \subfloat[Total emission intensity.]{
    \includegraphics[width=0.48\textwidth]{Images/Spectroscopy/Itot_flux2.png}
  }
  
  %\subfloat[UVB lines]{
  %  \includegraphics[width=0.48\textwidth]{Images/Spectroscopy/Irel_NO_flux.png}
  %}
  
  \subfloat[\ce{N_2} lines]{
    \includegraphics[width=0.48\textwidth]{Images/Spectroscopy/Irel_N2_flux2.png}
  }
  \hfill
  \subfloat[\ce{He} line]{
    \includegraphics[width=0.48\textwidth]{Images/Spectroscopy/Irel_He_flux2.png}
  }
  
  \subfloat[\ce{Ar} line for argon gas mixture]{
    \includegraphics[width=0.48\textwidth]{Images/Spectroscopy/Irel_Ar_flux2.png}
  }
\caption{Behaviour of total intensities (a) and relative intensities for selected portions of the spectrum (b-c-d), changing the composition of the gas. A flow of \SI{0.2}{\liter/\minute} corresponds to the $\num{10}\%$ of the total gas flow. Relative intensities take into consideration spectrometer's efficiency and total emission for each gas mixture.}
 \label{fig:Irel_flow}
\end{figure}


\begin{comment}
La determinazione delle specie reattive prodotte dal plasma a pressione atmosferica può essere effettuata tramite misure spettrometriche della sorgente in funzione su un bersaglio.
Gli effetti del trattamento al plasma si pensano dovuti alla presenza di ROS e RNS, quindi vengono raccolte misure nel range di lunghezze d'onda utili ad osservare le emissioni di molecole di \ce{OH} (\SI{305.00}-\SI{313.00}{\nano\meter}), \ce{N_{2}} (\SI{280.00}-\SI{500.00}{\nano\meter}) e \ce{NO} (\SI{220.0}-\SI{290.0}{\nano\meter}) (vedi articoli).

I prototipi di sorgente sviluppati permettono di variare il range dei parametri di funzionamento, in modo da modulare l'intensità del trattamento. Per verificare come cambia lo spettro in base alle diverse modalità di funzionamento, si osservano la variazione nell'intensità delle emissioni al cambiare di frequenza e tempo di apertura del circuito.

Si vuole osservare l'intensità relativa delle righe al variare del gas in ingresso, quindi la sorgente viene azionata con diverse miscele di gas. La misura standard viene effettuata con flusso di \ce{He}, nella solita modalità di funzionamento della sorgente. Vengono poi predisposte due ulteriori modalità di misura dove il gas viene fatto gorgogliare in una soluzione di acqua o di ammoniaca prima dell'inserimento all'uscita della sorgente, per arricchire i prodotti delle reazioni di ioni contenenti, rispettivamente, ossigeno o azoto.

A partire dalla forma delle righe di alcune specie molecolari, è inoltre possibile stimare la temperatura rotazionale alla quale avviene l'emissione misurata. Le emissioni dovute alla molecola \ce{OH} o \ce{N_2} sono varie righe dalle intensità variabili a seconda della temperatura delle molecole, misurando l'intensità relativa dei picchi si può ricavare la temperatura rotazionale delle molecole.

\section{Setup di acquisizione}

Per la sorgente vengono utilizzati sia il prototipo precedente, \textbf{prototipo 1} , sia il prototipo svilupato durante questo lavoro, \textbf{prototipo 2}.

Vengono inoltre provate tre diverse modalità di funzionamento della sorgente, variando frequenza e tempo di chiusura del circuito, mostrate in Tabella \ref{tab:setsorgente}.
Il flusso del gas di elio viene mantenuto pari a \SI{2}{\liter/\minute}.

\begin{table}
 \centering
 \begin{tabular}{cc}
 \toprule
 $f$ [\si{\kilo\hertz}]  &$\Delta t$ [\si{\micro\second}]\\
 \midrule
 5  &15\\
 10 &10\\
 15 &10\\
 \bottomrule
 \end{tabular}
 \caption{Parametri di funzionamento utilizzati per le diverse misure.}
 \label{tab:setsorgente}
\end{table}



\section{Presentazione ed analisi misure}
Per entrambe le sorgenti, l'analisi prevede il riconoscimento delle emissioni misurate, il confronto delle intensità variando distanza e parametri di funzionamento della sorgente, il confronto delle intensità variando il gas immesso, la stima delle temperature rotazionali delle molecole \ce{OH} e \ce{N_2}.

\subsection{Riconoscimento righe}
L'output dello spettrometro viene letto tramite routine IDL ed il riconoscimento dei picchi viene effettuato tramite la classe TSpectrum presente nelle librerie ROOT. Ogni misura viene confrontata con uno spettro di background preso con lo stesso tempo di acquisizione e sorgente spenta, riuscendo così ad escludere i picchi dovuti al fondo.
In figura \ref{fig:spettrotot} si vedono le principali righe estrapolate dalle acquisizioni, in particolare si vedono le righe relative ad \ce{NO}, \ce{OH}, \ce{H_{2}} e \ce{N_{2}}, tabulate in Tabella \ref{tab:spettrotot} .

\begin{figure}
\centering
\includegraphics[width=0.99\textwidth]{Immagini/spettrotot_unico_label_def.png}
\caption{Spettro acquisito con condizioni di misura standard ($f = \SI{5}{\kilo\hertz}$ e $\Delta t = \SI{16}{\micro\second}$), obiettivo puntato vicino l'uscita del gas dalla sorgente.}
\label{fig:app}
\end{figure}


L'acquisizione migliore, nella quale vengono riconosciuti più picchi, è quella mostrata in Figura \ref{fig:spettrotot}, corrispondente alla posizione 1 e condizioni standard di misura, $f = \SI{5}{\kilo\hertz}$.

\paragraph{}
Per verificare gli effetti di una diversa frequenza  nel funzionamento della sorgente vengono confrontate le intensità relative alle righe di \ce{OH} e \ce{N_2}, sommando i conteggi per le varie porzioni di spettro. Non vengono prese in considerazione le righe del gruppo relativo all'\ce{NO} in quanto troppo deboli. In Tabella \ref{tab:irel_1} i risultati, dove vengono confrontate i conteggi alle varie frequenze rispetto i conteggi ottenuti nelle condizioni standard di lavoro, $f = \SI{5}{\kilo\hertz}$. Si nota un calo evidente nei conteggi, crescente con l'aumentare della frequenza di lavoro.

Allo stesso modo viene variata la posizione dell'obbiettivo, dalla posizione 1, puntato all'uscita della sorgente, alla posizione 2, puntato all'uscita del bersaglio. I risultati sempre in Tabella \ref{tab:irel_1}, dove vengono presentati i conteggi nella posizione 2 rispetto i conteggi nella posizione 1. Viene trovato un effetto diverso sulle diverse specie, le emissioni di \ce{OH} diminuiscono molto, mentre quelle relative l'\ce{N_2} diminuiscono in maniera inferiore.

\begin{table}
 \centering
 \begin{tabular}{lcc}
 \toprule
                            &\ce{OH}  &\ce{N_2}\\
 \midrule
 f = \SI{5}{\kilo\hertz}    &1.00     &1.00\\
 f = \SI{10}{\kilo\hertz}    &0.95    &0.81\\
 f = \SI{15}{\kilo\hertz}    &0.62    &0.63\\
 \midrule
 posizione 1                &1.00        &1.00\\
 posizione 2                &0.10     &0.82\\
 \bottomrule
 \end{tabular}
 \caption{Intensità relative delle porzioni interessanti di spettro, al variare di frequenza e posizione, per il prototipo 1.}
 \label{tab:irel_1}
\end{table}


\subsection{\ce{He} + \ce{H_2O} e \ce{He} + \ce{NH_3}}
Vengono misurati gli spettri di emissione in condizioni di lavoro standard, $f = \SI{5}{\kilo\hertz}$, posizione 1, trattando il flusso di elio prima di inserirlo nella sorgente.
Per il prototipo 1, l'arricchimento di specie reattive dell'ossigeno o dell'azoto avviene facendo gorgogliare l'elio in una soluzione di acqua o ammoniaca, mantenendo un flusso in uscita dalla bombola sempre di \SI{2}{\liter/\minute}.
Le misure così acquisite presentano un'intensità molto minore per tutte le righe dello spettro e non vi sono variazioni rilevanti per le lunghezze d'onda interessate, cioè le emissioni relative ad \ce{OH} per la soluzione di acqua e relative ad \ce{N_2} per la soluzione di ammoniaca.

\subsection{Stima temperatura}
Lo spettro di emissione della molecola \ce{OH} è della forma presentata in Equazione \ref{eq:fitOH} (vedi articolo):
\begin{equation}
\centering
\begin{split}
&I_i (T) = I_{0,i} \exp{-\frac{E_n (T-T_0)}{T_0 T}} \\
&S_i(\lambda,T) = \frac{I_i(T)}{\sigma \sqrt{2\pi}} \exp{\frac{(\lambda - \lambda_i)^2}{2\sigma^2}}\\
&S(\lambda,T) = \sum_i S_i(\lambda,T)
\end{split}
\label{eq:fitOH}
\end{equation}

dove $I_i$ è l'intensità ad una determinata energia $E_n$ e $I_{0,i}$ è l'intensità misurata ad una temperatura di riferimento $T_0$. Le $S_i$ sono la convoluzione di $I_i$ con una distribuzione gaussiana nelle $\lambda$ e lo spettro finale $S$ sarà la somma degli spettri relativi la singola transizione.
Simulando diversi spettri nelle varie temperature, sarà possibile stimare la temperatura rotazionale degli ioni presenti nel gas.

La procedura consiste nel simulare diversi spettri $S(\lambda,T)$ come presentati in \ref{eq:fitOH}, calcolare gli scarti quadratici medi rispetto le misure e ricavare la temperatura dalla simulazione migliore. Tipicamente sono stati simulati spettri con $200$ temperature diverse, su un range di temperature possibili variabili, a seconda della misura considerata.
L'errore sulla stima della temperatura viene calcolato come media della differenza tra la temperatura minima che avesse uno scarto quadratico medio fino al $5\%$ superiore rispetto al minimo e la differenza della temperatura massima con le stesse condizioni.

In figura \ref{fig:fitT} sono presentati due esempi di fit delle emissioni interessate. In \ref{tab:Trot} vengono presentate le diverse temperature ottenute, compatibili tra loro.

\begin{figure}
 \centering
 \subfloat[][\ce{OH}]
  {\includegraphics[width=.8\textwidth]{Immagini/OHFit_f5t16elio.png}}
 \\
 \subfloat[][\ce{N_2}]
  {\includegraphics[width=.75\textwidth]{Immagini/N2rotFit_elio.png}}
 \caption{Fit delle emissioni con spettri simulati, misure prese con il prototipo 1, in condizioni standard, posizione 1.}
 \label{fig:fitT}
\end{figure}

\begin{table}
 \centering
 \begin{tabular}{cc}
 \toprule
            &T [\si{\kelvin}]\\
 \midrule
  \ce{OH}   &$336 \pm 30$\\
  \ce{N_2}  &$322 \pm 41$\\
 \bottomrule
 \end{tabular}
 \caption{Stima delle temperature di rotazione delle molecole.}
 \label{tab:Trot}
\end{table}
\end{comment}

\begin{comment}
La misura viene effettuata tramite uno spettrometro IsoPlane dalla lunghezza focale di \SI{320}{\milli\meter}, con tre diversi reticoli: \SI{150}, \SI{1200} e \SI{2400}{g/\milli\meter}, corrispondenti alle risoluzioni di ... .
La risoluzione maggiore viene utilizzata per acquisire le righe \ce{OH} e \ce{N_{2,\text{rot}}}, mentre per acquisire lo spettro totale vengono utilizzati i reticoli a piccola e media risoluzione.

Lo spettrometro è accoppiato ad una telecamera PIXIS di $2048 \times $ ... pixels quadrati dal lato di ... \si{\micro\meter}, con un massimo di \SI{65000} conteggi per il singolo canale.
La luce viene raccolta da una lente in quarzo di focale ... e diametro ... , portata da una fibra ottica dallo spessore di ... e lunghezza di ... e collegata all'entrata dello spettrometro.

Per l'acquisizione viene posizionata la sorgente in funzione a distanza di \SI{1}{\centi\metre} dal bersaglio in metallo (collegato a terra) con l'ottica focalizzata sul flusso di gas, come in foto \ref{fig:app}. Vengono distinte due posizioni dell'ottica:
\begin{itemize}
 \item posizione 1 = obiettivo sull'uscita della sorgente
 \item posizione 2 = obiettivo sul punto di impatto del plasma sulla sorgente, ad \SI{1}{\centi\meter} dall'uscita della sorgente
\end{itemize}
.

\begin{table}
\centering
 \begin{tabular}{ccc}
  \toprule
                            &$\lambda$ \text{[}\si{\nano\meter}\text{]} &\text{I [arb.u.]}\\
  \midrule
  \multirow{4}*{\ce{NO}}    &\num{236.31(24)}  &27\\
                            &\num{237.00(15)}  &26\\
                            &\num{247.02(5)}  &28\\
                            &\num{247.86(12)}  &27\\
  \midrule
  \multirow{2}*{\ce{OH}}    &\num{308.3(1)}  &106\\
                            &\num{309.1(1)}  &113\\
  \midrule
  \multirow{5}*{\ce{N_2}}   &\num{316.03(1)}  &381\\
                            &\num{337.11(1)}  &1000\\
                            &\num{357.77(1)}  &722\\
                            &\num{367.22(20)}  &58\\
                            &\num{371.12(4)}  &172\\
                            &\num{375.66(2)}  &232\\
                            &\num{380.64(2)}  &423\\
  \midrule
  \multirow{2}*{\ce{N_2^+}} &\num{391.50(2)}  &355\\
                            &\num{427.45(2)}  &180\\
  \midrule
  \ce{H_{\alpha}}           &\num{655.96(4)}  &113\\
  \midrule
  \multirow{2}*{\ce{He}}    &\num{586.94(5)}  &122\\
                            &\num{705.56(1)}  &649\\
  \midrule
  \ce{O}                    &\num{776.39(1)}  &393\\
  \bottomrule
 \end{tabular}
 \caption{Picchi rilevanti nello spettro di emissione del prototipo 1, condizioni di misura standard, posizione 1.}
 \label{tab:spettrotot}
\end{table}


Per ogni misura viene stabilito un tempo di acquisizione idoneo ad avere un numero ottimale di eventi, evitando la saturazione dei singoli canali. Una volta stabilito il tempo viene eseguita una misura di fondo, a sorgente spenta, e successivamente viene avviata la misura con sorgente attiva.
\end{comment}
